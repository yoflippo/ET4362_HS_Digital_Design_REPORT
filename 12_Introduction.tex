\section{Introduction} \label{sec:introduction} 

Almost all effort in the domain of computer engineering is put in increasing the data throughput and bandwidth. 
% Almost everyone if familiar with Moore's Law.
% Engineers are still actively trying to increase the performance of computers.
Chips are becoming increasingly larger (partially due to the increase of the number of cores).
So more data can be processed in the same amount of time.
Data has to be shared between cores and other components of a CPU.
This is were High Speed Interfaces come in play. 
However, data not only has to be shared between cores but also between other components of a computer system.
To get the most out of a computer system all parts should be aligned, so no bottlenecks are present.


In this review 4 papers are summarized regarding \textit{High Speed Interfaces}. 

% Motive: Statement indicating why the research was done (e.g. a gap in
% knowledge, contradictory results). The motive leads to the objective.
% • Objective: Statement about what the authors want to know. The
% objective may be formulated as a research question, a research aim, or a
% hypothesis that needs to be tested.
% • Main conclusion: Statement about the main outcome of the research. The
% main conclusion is closely connected to the objective. It answers the
% research question, it says whether the research aim was achieved, or it
% states whether the hypothesis was supported by evidence.



% %Motive
% Most research tends to present a singular perspective on the optimization of \ac{ra}.
% Combining the results of different innovations has not been done so far and could lead to further improvements.
% % Objective
% Therefore the objective of this review is to try to find feasible options to combine the proposed techniques used in the reconfigurable architecture domain.

% % Structural overview
% The remainder of this review is organized as follows.
% \Cref{sec:background} elaborates on the background of reconfigurable architectures. 
% Most recent techniques will be discussed in \cref{sec:sota}.
% In \cref{sec:possible_combs}, possible combinations of these techniques are discussed. 
% Finally, the most feasible combination of techniques is proposed in \cref{sec:conclusion}.

