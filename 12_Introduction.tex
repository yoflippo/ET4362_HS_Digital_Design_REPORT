\section{Introduction} \label{sec:introduction} 

Almost all effort in the domain of computer engineering is put in increasing the data throughput. 
% Chips are becoming increasingly larger (partially due to the increase of the number of cores).
% So more data can be processed in the same amount of time.

Computer systems consists of different parts. 
Most bottlenecks of computer systems exist within the transitions or crossings of these parts.
E.g. Data has to be shared between the processor cores, cache, RAM, GPU, etc.
High speed interfaces enable designers to alleviate or circumvent bottlenecks.
However, data not only has to be shared between cores but also between other components of a computer system.
To get the most out of a computer systems all parts should be structurally aligned.

In this review 4 papers are covered regarding the theme \textit{High Speed Interfaces}. 
From a high level perspective the papers revolve around increasing the bandwidth or throughput of a computer system which would otherwise lead to bottlenecks and on how to diminish them.
The central question being: in what way can the techniques described be combined to benefit the on-chip and off-chip performance of a computer system?

To give an overview of the covered papers:
The first paper (\cite{steenkiste1997high}) revolves around a high-speed network interface for distributed-memory systems.
The second paper (\cite{agrawal20098}) discusses the creation of a 5 Gb/s parallel receiver with collaborative timing recovery.
The third paper (\cite{schinkel2009low}) talks about a high-speed transceiver for network-on-chip communication.
The fourth paper (\cite{zhang2009high}) revolves around high performance on-chip differential signaling using passive compensation.
The area of high speed interfaces requires knowledge about high speed analog and digital design.



% Motive: Statement indicating why the research was done (e.g. a gap in
% knowledge, contradictory results). The motive leads to the objective.
% • Objective: Statement about what the authors want to know. The
% objective may be formulated as a research question, a research aim, or a
% hypothesis that needs to be tested.
% • Main conclusion: Statement about the main outcome of the research. The
% main conclusion is closely connected to the objective. It answers the
% research question, it says whether the research aim was achieved, or it
% states whether the hypothesis was supported by evidence.



% %Motive
% Most research tends to present a singular perspective on the optimization of \ac{ra}.
% Combining the results of different innovations has not been done so far and could lead to further improvements.
% % Objective
% Therefore the objective of this review is to try to find feasible options to combine the proposed techniques used in the reconfigurable architecture domain.

% % Structural overview
% The remainder of this review is organized as follows.
% \Cref{sec:background} elaborates on the background of reconfigurable architectures. 
% Most recent techniques will be discussed in \cref{sec:sota}.
% In \cref{sec:possible_combs}, possible combinations of these techniques are discussed. 
% Finally, the most feasible combination of techniques is proposed in \cref{sec:conclusion}.