
%%%%%%%%%%%%%%%%%%%%%%%%%%%  5  %%%%%%%%%%%%%%%%%%%%%%%%%%% 
\subsection{A high-speed network interface for distributed-memory systems: architecture and applications \cite{steenkiste1997high}} \label{ss:steenkiste1997high}
%%%%%%%%%%%%%%%%%%%%%%%%%%%  5  %%%%%%%%%%%%%%%%%%%%%%%%%%% 

% Background of study, is about the research on which you base your study problem, why are you doing it. 
% A review paper gives an overview of research dealing with a specific topic. 
% \lettrine{O}{ne} of the first early mentions of \ac{ra} was in papers by Iseli and Baxter \cite{iseli1993spyder, baxter1999icarus}. 
% However, these implementations were limited in the reconfigurability of functional units and the issue width. % Filler, can be removed.
% Different closed source alternatives were developed as well \cite{Vassiliadis2004Molen,Hauck20014Chimaera}.
% Some years later the first usable open source platform was developed by Wong in 2008 \cite{wong2008rho}. 
% Before that \ac{ra} techniques were closed source and limited due to the state of technology scaling i.e. the number of transistors on a chip.

% % Some extra explanation, could be trimmed in a future version
% % \ac{ra} is closely related to the issue width of an architecture. % Filler, can be removed
% % The issue width of an architecture determines how many instructions can be executed at the same time. . % Filler, can be removed
% \acsp{ra} come in two types: \acc{sp} and \acc{vliw}.
% The difference between \ac{sp} and \ac{vliw} lies in the way the hardware architecture is approached by software.
% The \ac{sp} architecture contains complex hardware to analyze the instructions (software) and to prevent dependencies.
% The \ac{vliw} architecture is associated to a compiler.
% The compiler processes a program in a bundle of instructions optimized for a certain \ac{vliw} architecture.
% This allows for a relatively simple architecture and low power consumption \cite{hubener2014coreva}. 
% On the other hand it heavily depends on a compiler which makes the process of developing a computing platform more complex.
% This review focuses on \ac{vliw}.

% An open source \ac{ra} called \acr{rvex} was introduced back in 2008 by Wong\cite{wong2008rho}. 
% This processor can adjust its core architecture with the adjustment of a few lines \acr{vhdl} code after which it can be re-synthesized and implemented on a \acr{fpga}, a programmable chip. 
% In doing so, creating the possibility to optimize the core for a specific application. 
% The upsurge of \accs{hra} led to the creation of tools to optimize the design exploration of these systems (\cite{jordans2013exploring}).

% % A sentence to try to explain the focus of papers of interest, to make it explicitly
% Generally speaking \ac{ra} research focuses on the improvement of performance and energy consumption \cite{jordans2013exploring,giraldo2015energysavingdynamicrc}.  

% \section{State of the Art Techniques}
% \label{sec:sota}
% \lettrine{T}{he} following paragraphs describe state of the art techniques regarding improved implementations of \acsp{ra}.

% \subsection{Abstraction level}
% A problem with reconfigurable hardware is the complexity and the increasing time to market \cite{wold2014relocatable}.
% To diminish this problem Wold suggests to raise the abstraction level of the \ac{ra} design process. 
% Development time can become more efficient by combining an \ac{os} called ReconOS and hardware threads.
% This is supported by the development of open source tools that simplify the design of \ac{ra} \cite{Sohanghpurwala2011openpr}.

% \subsection{Energy consumption}
% A big body of knowledge supports the fact that the cache of a system consumes most (approximately 50\%) of the energy of a \ac{ra} \cite{malik2000lowpower,yang2002energyefficientcache,zhang2003configurablecache}. 

% Anjam et al. changed the instruction cache (I-cache) of a processor simultaneously with the reconfiguration of the platform \cite{anjam2014Simultaneous}. 
% By adjusting the cache size simultaneously with the \ac{ra} configuration the \acc{edp} (see \cref{eq:edp}) can be reduced up to 37\% \cite{anjam2014Simultaneous}. 
% As can be seen in \cref{fig:2014AnJamsimultaneous} the performance of two benchmarks highly depends on the cache size and not the issue width.

% \begin{equation} \label{eq:edp}
% EDP = Energy \cdot {Cycles} 
% \end{equation}

% \begin{figure}[H]
% 	\centering
% 	\includegraphics[width=1\linewidth]{figures/2014AnJamsimultaneous.png}
% 	\caption{ The effect of issue width and cache size on EDP Energy and speedup considering two benchmarks: Patrica and Pocsag. Source: \cite{anjam2014Simultaneous} p.7} 
%     \label{fig:2014AnJamsimultaneous}
% \end{figure}

% \subsection{Optimize resource utilization}
% Often, a set of instructions belongs to particular part of the program code that is repeated a lot.
% A \ac{rra} can adjust its architecture to the needs of the executed set of (repeated) instructions.
% This could lead to underused hardware in an \ac{fpga} \cite{brandon2015sparse}.
% Brandon proposes a way to address this resource-utilization problem by combining unused functional units to form extra processor cores which can execute more threads.
% Hence optimizing the overall throughput and resource usage. % niet helemaal waar, eigenlijk gaat de paper hoofdzakelijk over de stop bit. Het voorgaande wordt aangestipt.

% The previously described way of dealing with resource-utilization does not lead to optimal usage of \acrp{bram} in an \ac{fpga} \cite{hoozemans2015Contexts}.
% For instance, a \ac{bram} has a 512 bytes size, but an architecture can only use 64 out of the 512 bytes.
% By using the remaining \ac{bram} space multiple contexts can be created.
% Hence, this reduces the time of a context switch because the pipeline does not have to be flushed (i.e. it can be saved in the remaining available \ac{bram} space).

% Srinivasan, mentions the overhead problem when using a \acc{ha} (i.e. an architecture with an asymmetric issue width, e.g. 2 and 4, 2 and 8) \cite{Srinivasan2016EHW}.
% Frequently switching from the architecture configuration leads to unnecessary overhead.
% This problem is countered by proposing a limited 4 core mode and a small number of performance indicators.

% \subsection{Adjustable cache}
% Hoozemans, proposes a dynamically adjustable cache for the \ac{rvex} processor \cite{hoozemans2017cache}. 
% This way each time the processor switches context, the cache is adjusted to the new context. 
% The adjustable cache is implemented with the \ac{rvex} instruction set. 
% This instruction set supports the identification of private and shared data in caches. 
% Therefore creating the possibility to only broadcast shared data over the data-bus while only updating the private data to the main memory when the process is finished.
% The results obtained from this implementation are more or less identical to other shared cache mechanisms while reducing the number of broad casted cache requests by 21\%.

% Inherent application ILP (Instruction Level Parallelism) creates unused resources and loss in performance as stated by Brandon \cite{brandon2017}. 
% Brandon, proposes a technique to increase the continuous use of resources and hereby increasing the overall performance no matter what application needs to be executed.
% The designed dynamic (polymorphic) implementation of the \ac{vliw} processor shows an increase of 7\% in performance and 5\% less area used compared to a processor that uses fixed cores \cite{brandon2017}.



% \section{Possible combinations} \label{sec:possible_combs}
% \lettrine{I}{n} the previous section different techniques were covered that improve the performance of a \ac{ha} or \ac{ra}.
% To sum up:
% \begin{itemize}
% \item Raising the abstraction level of the \ac{ra} design process \cite{wold2014relocatable}
% \item Adjusting the instruction cache simultaneously with architecture reconfiguration \cite{anjam2014Simultaneous}
% \item Re-using functional units to form an extra processor core \cite{brandon2015sparse}
% \item Using the remaining \ac{bram} space to create multiple contexts \cite{hoozemans2015Contexts}
% \item Diminishing the overhead of too many architecture reconfigurations by a limited number of modes and some simple performance counters \cite{Srinivasan2016EHW}
% \item Using a dynamically adjustable cache \cite{hoozemans2017cache, brandon2017}
% \end{itemize}

% The approach of Srinivasan \cite{Srinivasan2016EHW} could be combined with all the other previously mentioned techniques.
% Because all above mentioned papers use an \ac{fpga} as their implementation platform.
% The proposed limitation of the number of cores prevents too frequent reconfigurations of an architecture.

% Furthermore, all described techniques could benefit from using the remaining \ac{bram} space.
% Not using a \ac{fpga} to its full extend is a wasted of technology.
% This will improve the time it takes to switch from a context significantly (make it almost instantaneously).

% Combining the work of Anjam \cite{anjam2014Simultaneous} and Hoozemans \cite{hoozemans2017cache} looks to be promising in the sense that they are complementary.
% However, limiting the reconfigurability, as proposed by Srinivasan, could diminish the effects of cache adjustment. 
% The same goes for the reuse of functional units as proposed by Brandon \cite{brandon2015sparse} and Srinivasan \cite{Srinivasan2016EHW}. 

% Raising the abstraction level of the design process could be useful in a later stage when software should be adapted for the final architecture.
% So this technology improvement is not a feasible combination of the previously mentioned improvements.
% This solution focuses on the environment in which \acsp{ra} are developed, not the direct improvement of \acsp{ra}.

% To conclude, limiting the number of architectural modes and use of the remaining \ac{bram} space in combination with a reconfigurable cache we expect to achieve improved results. 
% An apparent difficulty lies in the design choice of combining a limited architectural processor and dynamic reconfigurability of the cache.
% Nonetheless, it should theoretically lead to a better performance of a \ac{ra}. 
