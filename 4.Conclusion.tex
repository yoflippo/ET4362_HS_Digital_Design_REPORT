\section{Conclusion} \label{sec:conclusion}


% Conclusions and future prospects
% are clearly linked to the research question in the introduction, indicating to what extent the research question has been answered
% follow logically from all the previous material 
% are preferably followed by recommendations for further research (future prospects)word limit is respected (+/- 200)

% In the conclusion, you should:

% - Summarize major contributions of significant studies and articles to the body of knowledge under review, maintaining the focus established in the introduction.
% - Evaluate the current "state of the art" for the body of knowledge reviewed, pointing out major methodological flaws or gaps in research, inconsistencies in theory and findings, and areas or issues pertinent to future study.
% - Conclude by providing some insight into the relationship between the central topic of the literature review and a larger area of study such as a discipline, a scientific endeavor, or a profession.

\section{Remarks}
From \cite{agrawal20098} the importance of the \ac{pll} and the \ac{dll} was grasped. 
This is a subject that could be engaged more in the lecture slides of this course.
The paper from \cite{agrawal20098} was pleasant to review but relatively hard to summarize without diving into too much details.