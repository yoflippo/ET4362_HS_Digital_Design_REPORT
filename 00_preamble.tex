\documentclass[conference]{IEEEtran}

\usepackage[pdftex]{graphicx}
\usepackage{verbatim}
\usepackage{float}
\usepackage{booktabs}
\usepackage[colorlinks=true,linkcolor=black,urlcolor=blue, citecolor=blue]{hyperref} %Adds hyperlinks in pdf version
\usepackage[open,numbered]{bookmark}%Adds bookmarks in pdf version with opened sublevels
\usepackage[T1]{fontenc} % Use 8-bit encoding that has 256 glyphs
\usepackage{fourier} % Use the Adobe Utopia font for the document - comment this line to return to the LaTeX default
\usepackage[english]{babel} % English language/hyphenation
\usepackage{amsmath,amsfonts,amsthm} % Math packages
\usepackage{lipsum} % Used for inserting dummy 'Lorem ipsum' text 
\usepackage{charter} %font die HR mooi vindt
\usepackage{fancyhdr} % Custom headers and footers
% \usepackage{listings}

\pagestyle{fancyplain} % Makes all pages in the document conform to the custom headers and footers
\fancyhead{} % No page header - if you want one, create it in the same way as the footers below
\fancyfoot[L]{} % Empty left footer
\fancyfoot[C]{} % Empty center footer
\fancyfoot[R]{\thepage} % Page numbering for right footer
\renewcommand{\headrulewidth}{0pt} % Remove header underlines
\renewcommand{\footrulewidth}{0pt} % Remove footer underlines
\setlength{\headheight}{13.6pt} % Customize the height of the header
\numberwithin{equation}{section} % Number equations within sections
% \numberwithin{figure}{section} % Number figures within sections
\numberwithin{table}{section} % Number tables within sections
%\setlength\parindent{0pt} % Removes all indentation from paragraphs

\usepackage{tabu} % For tables (TvE)
\usepackage{cleveref} % For clever references (\cref, TvE)
\usepackage[sharp]{easylist} % MS: easy listing
\newcommand{\horrule}[1]{\rule{\linewidth}{#1}} % Create horizontal rule command with 1 argument of height
\usepackage[usenames,dvipsnames,svgnames,table]{xcolor} %TvE: text coloring
\usepackage[utf8]{inputenc}
\usepackage{lettrine}
\usepackage[hang]{footmisc} %MS: footnote alignment, https://tex.stackexchange.com/questions/126877/how-can-i-align-a-multiple-line-footnote-text-right-to-the-footnote-mark
\setlength\footnotemargin{6pt}

%To create a list of abbreviations
\usepackage{acro}
\acsetup{first-style=short, hyperref = true}
% \newcommand*\aclabelfont[1]{\textbf{\acsfont{#1}}}

\DeclareAcronym{synclink}{
  short = SOUS,
  long  = Source Synchronous Links / Forwarded clock link,
  class = abbrev
}
\DeclareAcronym{embeddedlink}{
  short = SERS,
  long  = Embedded clock Link / Ensemble serial clock link,
  class = abbrev
}
\DeclareAcronym{coltimrec}{
    short = CTR,
    long = Collaborative Timing Recovery,
    class = abbrev
}
\DeclareAcronym{cdr}{
    short = CDR,
    long = Clock Data Recovery Loop,
    class = abbrev
}
\DeclareAcronym{gtr}{
    short = GTR,
    long = Global Timing Recovery,
    class = abbrev
}
\DeclareAcronym{noc}{
    short = NoC,
    long = Network on Chip,
    class = abbrev
}
\DeclareAcronym{bw}{
    short = BW,
    long = Bandwidth,
    class = abbrev
}
\DeclareAcronym{dll}{
    short = \textbf{D}LL,
    long = Delayed Locked Loop,
    class = abbrev
}
\DeclareAcronym{pll}{
    short = \textbf{P}LL,
    long = Phase Locked Loop,
    class = abbrev
}
\DeclareAcronym{otl}{
    short = T-Line,
    long = On-chip tranmission line,
    class = abbrev
}
\DeclareAcronym{isi}{
    short = ISI,
    long = intersymbol interference,
    class = abbrev
}
\DeclareAcronym{sqp}{
    short = SQP,
    long = Sequential Quadratic Programming,
    class = abbrev
}



\newcommand{\acc}[1]{\acl{#1} (\ac{#1})}
\newcommand{\Acc}[1]{\Acl{#1} (\ac{#1})}
\newcommand{\accs}[1]{\acl{#1}s (\ac{#1})}
\newcommand{\Accs}[1]{\Acl{#1}s (\ac{#1})}
% \newcommand{\accp}[1]{\aclp{#1} (\ac{#1})}
% \newcommand{\acr}[1]{\ac{#1} (\acl{#1})}
% \newcommand{\acrp}[1]{\acsp{#1} (\acl{#1})}

%% COMMANDS AND PACKAGES FOR REVIEW PAPERS FOR HS DIGITAL EMBEDDED SYSTEMS
\newcommand{\objective}{\textit{Objective: }}
\newcommand{\motive}{ \textit{Motive: }}
\newcommand{\summary}{\textit{Summary: }}
\usepackage{cancel}